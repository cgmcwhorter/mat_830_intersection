% !TEX root = ../../intersection_theory.tex

\newpage
\section{Cones}

Recall vector bundles and projective bundles where $\spec$ and $\Proj$ of the graded sheaf of $\co_X$-algebras $\Sym(\cE)$, $\cE$ a locally free sheaf of finite rank. We also said you can do that for more general sheaves of graded $\co_X$-algebras. We called them cones. Recall how we defined Segre classes of vector bundles: $E$ rank $e+1$ on $X$ a vector bundle and $p: P(E) \to X$. For $\alpha \in A_kX$, we defined $s_i(E) \cap \alpha= p_*(c_1(\co(1))^{e+i} \cap p_*\alpha)$. Why cant we do the same thing for a cone? $\pi: C \to X$ and $p:P(C) \to X$ for $\alpha \in A_kX$.
	\[
	s_i(C) \cap \alpha = p_*(c_1(\co(1))^{e+i} \cap p^* \alpha
	\]
There are two good reasons. First, what is $e$? Second, $p$ might not be flat so that $p^*\alpha$ might not be defined. Let $C$ be a cone over $X$ and $C=\spec(S^\cdot)$. Let $S^\cdot$ be the sheaf of graded $\co_X$-algebras and $\co_X \to S^0$ be surjective. Now $S^1$ is coherent and $S^\cdot$ is generated by $S^1$. So $P(C \oplus 1)=\Proj(S^\cdot[z])$. $q: P(C \oplus 1) \to X$ and you have a $\co(1)$ on $P(C \oplus 1)$.


\begin{dfn}[Segre Class]
The Segre class of $C$, denoted $s(C)$, is the class in $A_*X$ defined by the formula:
	\[
	s(C)= q_*\left(\sum_{i \geq 0} c_1(\co(1))^i \cap [P(C \oplus 1)] \right)
	\]
Note that the Segre class of a cone is an element of the Chow group of $X$. It is not a homomorphism on the Chow group of $X$ as the Segre classes of vector bundles were. 
\end{dfn}

As vector bundles were the elements of cohomology, cones are elements of the homology. Thus, it does not make any sense to define $c(C)=s(C)^{-1}$ as was done in vector bundles. Things like $s_1^2$, $s_1s_2$ do not make sense. We do not know how to intersect elements of $A_*$ with each other.

\begin{prop}
\begin{enumerate}[(a)]
\item If $E$ is a vector bundle on $X$, then $s(E)=c(E)^{-1} \cap [X]$, where $c(E)=1+c_1(E)+c_2(E)+\cdots+c_r(E)$ is the total Chern class of $E$, $r$ is the rank of $E$. 
\item 
\end{enumerate}
\end{prop}

Think of $s(E)$ as the cone Segre class. 

\begin{prop} \hfill
\begin{enumerate}[(a)]
\item If $E$ is a vector bundle on $X$, then
	\[
	s(E)= c(E)^{-1} \cap [X]
	\]
where $c(E)=1+c_1(E)+\cdots+c_r(E)$ is the total Chern class of $E$ and $r=\rank E$.
\item Let $c_1,\ldots,c_t$ be the irreducible components of $C$, $m_i$ the geometric multiplicty of $c_i$ in $C$. Then
	\[
	s(C)= \sum_{i=1}^t m_i s(C_i)
	\]
\end{enumerate}
\end{prop}

\pf For (a), since $[P(E \oplus 1)]=q^*[X]$, the above definition of $s(E)$ agrees with the definition of $c(E \oplus 1)^{-1} \cap [X]$, $q: P(E \oplus 1) \to X$, $s(c)=q_*(\sum_{i \geq 0} c_1(\co(1))^i \cap [P(C \oplus 1)]$, $c(E \oplus 1)^{-1})=s(E)$.
	\[
	s_i(E \oplus 1) \cap [X]= q_*(c_1(\co(1))^{e+1+i} \cap [P(E \oplus 1)]
	\]
It looks like $s(c)$ has more terms than $s_i(E\oplus 1) \cap [X]$. But $[P(E \oplus 1)]$ has dimension $\dim X+e+1$ and we know $\rank E=e+1$. Each time you intersect with $c_1(\co(1))$, the dimension goes down by one. So $q_*$ preserves dimension. As for $A_iX$, $A_iX=0$ for $i>\dim X$. Until you intersect with $\co(1)$ $(e+1)$-times, the push forward must be 0 because the dimension is too large. We have $s(E)=c(E \oplus 1)^{-1} \cap [X]$. There is an exact sequence
	\[
	0 \ma{} \co \ma{} E \oplus 1 \ma{} E \ma{} 0
	\]
Whitney sum: $c_t(\co)=1$. \qed \\


\subsection{Segre Class of a Subscheme \& Differentials}


First, we talk about tangent and normal bundles. If $X$ is nonsingular and $Y \subset X$ is a nonsingular subscheme, then
	\[
	0 \ma{} T_Y \ma{} T_X\big|_Y \ma{} N_{Y/X} \ma{} 0
	\]
where $N_{Y/X}$ is the normal bundle of $Y$ in $X$. Let $A$ be a ring (commutative with identity as always), $B$ be an $A$-algebra, and $M$ be a $B$-module. 

\begin{dfn}[Derivation]
An $A$-derivation of $B$ into $M$ is a map $d: B \to M$ such that 
\begin{enumerate}[(i)]
\item $d$ is additive
\item $d(bb')= bdb'+b'db$
\item $da=0$ for all $a \in A$.
\end{enumerate}
\end{dfn}

\begin{dfn}
We define the module of relative differential forms of $B$ over $A$ to be the $B$-module $\Omega_{B/A}$ together with a $A$-derivation $d: B \to \Omega_{B/A}$ which satisfies the following universal property: for any $B$-module $M$ and for any $A$-derivation $d': B \to M$, there exists a unique $B$-module homomorphism $f: \Omega_{B/A} \to M$ such that $d'= f \circ d$.
\end{dfn}

\begin{prop}
Let $B$ be an $A$-algebra. Let $f: B \otimes_A B$ be the ``diagonal'' homomorphism defined by $f(b \otimes b')=bb'$ and let $I=\ker f$. Consider $B \otimes_A B$ as a $B$-module by multiplication on the left. Then $I/I^2$ inherits a structure of a $B$-module. Define a map $d: B \to I/I^2$ by $db= 1 \otimes b - b\otimes 1$. Then $\langle I/I^2,d\rangle$ is a module of relative differentials for $B/A$.
\end{prop}

Let $f: X \to Y$ be a morphism of schemes. We consider the diagonal morphism $\Delta: X \to X \times_Y X$. Now $\Delta$ gives an isomorphism of $X$ onto its image $\Delta(X)$, which is a locally closed subscheme of $X \times_Y X$, i.e. a closed subscheme of an open subset $W$ of $X \times_Y X$.

\begin{dfn}[Sheaf of Relative Differentials]
Let $\ci$ be the sheaf of ideals of $\Delta(X)$ in $W$. Then we define the sheaf of relative differentials of $X$ over $Y$ to be the sheaf $\Omega_{X/Y}=\Delta^*(\ci/\ci^2)$ on $X$.
\end{dfn}


\begin{rem}
This locally on affine patches looks like the proposition above.
\end{rem}

\begin{prop}
Let $f: X \to Y$ be a morphism and let $Z$ be a closed subscheme of $X$ with ideal sheaf $\ci$. Then there is an exact sequence of sheaves on $Z$,
	\[
	\ci/\ci^2 \ma{\delta} \Omega_{X/Y} \otimes \co_Z \ma{} \Omega_{Z/Y} \ma{} 0
	\]
\end{prop}


\begin{dfn}
An (abstract) variety $X$ over an algebraically closed field $k$ is nonsingular if and only if all its local rings are regular local rings.
\end{dfn}

\begin{thmm}
Let $X$ be an irreducible separated scheme of finite type over an algebraically closed field $k$. Then $\Omega_{X/k}$ is a locally free sheaf (of rank $n=\dim X$) if and only if $X$ is a nonsingular variety over $k$.
\end{thmm}


\begin{thmm}
Let $A$ be a ring, $Y=\spec A$, and $X=\P_A^n$. Then there is an exact sequence of sheaves on $X$
	\[
	0 \ma{} \Omega_{X/Y} \ma{} \co_X(-1)^{n+1} \ma{} \co_X \ma{} 0
	\]
[The exponent $n+1$ in the middle means a direct sum of $n+1$ copies of $\co_X(-1)$.]
\end{thmm}

When $A=k$ is algebraically closed and $X$ is nonsingular, this allows us to compute the Chern classes of $\Omega_{X/k}$. 

\begin{thmm}
Let $X$ be a nonsingular variety over $k$. Let $Y \subseteq X$ be an irreducible closed subscheme defined by a sheaf of ideals $\ci$. Then $Y$ is nonsingular if and only if 
\begin{enumerate}[(i)]
\item $\Omega_{Y/k}$ is locally free
\item the sequence 
	\[
	0\ma{} \ci/\ci^2 \ma{\delta} \Omega_{X/Y} \otimes \co_Z \ma{} \Omega_{Z/Y} \ma{} 0
	\]
is exact.
\end{enumerate}
Furthermore, in this case, $\ci$ is locally generated by $r=\codim(Y,X)$ elements and $\ci/\ci^2$ is a locally free sheaf of rank $r$ on $Y$.
\end{thmm}

\begin{dfn}[Tangent Sheaf]
Let $X$ be a nonsingular variety over $k$. We define the tangent sheaf of $X$ to be $\ct_X=\Hom(\Omega_{X/k}, \co_X)$. It is a locally free sheaf of rank $n=\dim X$. If $X$ is projective and nonsingular, we define the geometric genus of $X$ to be $p_g=\dim_k \Gamma(X,w_X)$. We define the canonical sheaf of $X$ to be $w_X=\wedge^n \omega_{X/k}$, the $n$th exterior power of the sheaf of differentials, where $n=\dim X$. This is an invertible sheaf on $X$.
\end{dfn}


\begin{dfn}[(Co)Normal Sheaf]
Let $Y$ be a nonsingular subvariety of a nonsingular variety $X$ over $k$. The locally free sheaf $\ci/\ci^2$ we call the conormal sheaf of $Y$ in $X$. Its dual $N_{Y/X}:=\Hom(\ci/\ci^2.\co_Y)$ is called the normal sheaf of $Y$ in $X$. It is locally free of rank $r=\codim(Y,X)$. 
\end{dfn}

To see this is as it should be, take the dual of the exact sequence of the theorem:
	\[
	0 \ma{} \ct_Y \ma{} \ct_X \otimes \co_Y \ma{} N_{Y/X} \ma{} 0
	\]
The sheaf $\ci/\ci^2$ makes sense even in the singular case.


Let $X$ be a closed subscheme of a scheme $Y$. Let $C=C_XY$ be the normal cone to $X$ in $Y$:
	\[
	C=\spec\left(\sum_{n=0}^\infty \ci^n/\ci^{n+1}\right)
	\]
where $\ci$ is the ideal sheaf defining $X$ in $Y$. The Segre class of $X$ in $Y$, denoted $s(X,Y)$, is defined to be the Segre class of the normal cone $C$:
	\[
	s(X,Y)=s(C_XY) \in A_*Y.
	\]
Suppose $X,Y$ are nonsingular so you have the conormal sheaf $\ci/\ci^2$ being locally free of rank $r=\codim(X,Y)$. How do you get the vector bundle associated to $\ci/\ci^2$?
	\[
	\spec\left(\bigoplus_{n=0}^\infty \Sym^n \ci/\ci^2\right)
	\]
is the vector bundle for the normal bundle. In the nonsingular case
	\[
	\begin{split}
	\bigoplus_{n=0}^\infty \Sym^n \ci/\ci^2 &\cong \sum_{n=0}^\infty \ci^n/\ci^{n+1} \\
	\ci^n/\ci^{n+1} \otimes \ci/\ci^2 &\cong \ci^{n+1}/\ci^{n+2}
	\end{split}
	\]
Think of the Commutative Algebra cse: if $(R,\fm)$ is a regular local ring, $\dim d$
	\[
	\bigoplus_{i=0}^\infty \fm^i/\fm^{i+1}= k[x_1,\ldots,x_d]
	\]
where $x_i$ is a system of local parameters and $\fm/\fm^2=\text{span}\{x_1,\ldots,x_d\}$. In the nonsingular case, the normal cone is the normal bundle. 

\begin{dfn}[Multiplicity along a Subvariety]
For an irreducible subvariety $X$ of a variety $Y$, the coefficient of $[X]$ in the class $s(X,Y)$ is called the multiplicity of $Y$ along $X$, or the algebraic multiplicity of $X$ on $Y$ and is denoted $e_XY$. When $X=P$ is a (closed) point, $e_PY$ is called the multiplicity of $Y$ at $P$.
\end{dfn}

\begin{ex}
\begin{enumerate}[(i)]
\item Let $P$ be a nonsingular point of a variety $Y$. $\co_{P,Y}$ is a regular local ring. Say $\dim Y=d$. Then 
	\[
	\bigoplus_{i=0}^\infty \fm^i/\fm^{i+1} \cong k[x_1,\ldots,x_d]
	\]
and $\spec$ of this is $\A^d$ which is what one would think the tangent space should be. A point has no tangent space
	\[
	0 \ma{} \ct_X \ma{} \ct_Y \ma{} N_{X/Y} \ma{} 0
	\]
and when $X$ is a (closed) point then $\ct_X=0$.
	\[
	s(C)=q_*\left(\sum_{i \geq 0} c_1(\co(1))^i \cap P[C \oplus 1]\right)
	\]
Now $C \cong \A^d$ and $P(C \oplus 1) \cong \P^d$. $\co(1)$ corresponds to a hyperplane. The proper pushforward $\P^d \ma{q} P$ preserves dimension. So one does not obtain anything nonzero until you have something in dimension 0. If one intersects a nonempty collection of hyperplanes until one obtains a finite nonempty set, how many points are in that finite set? One! The coefficient of $[P]$ in $s(P,Y)=1$. A nonsingular point should have multiplicity 1 and it does.

\item Take $Y=Z(y^2-x^3) \subseteq \A^2$, the cuspidal cubic, and $X=(0,0)$. Let $\fm=(x,y)$ the maximal ideal of $P$ in $\A^2$.
	\[
	\bigoplus_{i=0}^\infty \fm^i/\fm^{i+1} \cong k[x,y]
	\]
as $\fm^0/\fm=k$, $\fm/\fm^2=(x,y)$, $\fm^2/\fm^3=(x^2,xy,y^2)$, $\fm^3/\fm^4=(x^3,x^2y,xy^2,y^3)$, $\cdots$. Now we mod out by $y^2-x^3$, i.e. set $y^2=x^3$. However in $\fm^2/\fm^3$, $x^3=0$. But then $y^2=0$. Now $k[x,y]/(y^2)$ is the double $x$--axis. $\co(1)$ is the point and $P(C \oplus 1)=\P^1$. When you push down, we obtain that the point has multiplicity 2.

\item Let $Y=Z(y^2-x^2(x+1)) \subseteq \A^2$, the nodal cubic, and $X=(0,0)$. Now $\fm=(x,y)$ and 	\[
	\bigoplus_{i=0}^\infty \fm^i/\fm^{i+1} \cong k[x,y]
	\]
Now $y^2=x^3+x^2$ so that in $\fm^2/\fm^3$, $x^3=0$. Then $y=x^2$ and $y^2=x^2=0$ so that $(y-x)(x+y)=0$. Then the tangent cone is these two lines. $\co(1)$ is going to be one point on each and $P(C \oplus 1)$ and we get two $\P^1$'s and thus the point has multiplicity two. 
\end{enumerate}
\end{ex}




\section{Deformation to the Normal Cone}

Take the example of $y^2=x^2(x+t)$ and let $t \to 0$. The cone of the nodal cubic degenerates to the cone of the cuspidal cubic. Think of the two lines of the cone join to be one double line. Let $X$ be a closed subscheme of a scheme $Y$ and let $C=C_XY$ be the normal cone to $X$ in $Y$. We will construct a scheme $M=M_XY$ together with a closed imbedding of $X \times \P^1$ in $M$ and a flat morphism $\varrho: M \to \P^1$ so that
	\[
	\begin{tikzcd}
	X \times \P^1 \arrow[hook]{r} \arrow[swap]{dr}{p^r} & M \arrow{d}{\varrho} \\
	& \P^1
	\end{tikzcd}
	\]
commutes and such that 
\begin{enumerate}[(i)]
\item Over $\P^1 \setminus \{\infty\}=\A^1$, $\varrho^{-1}(\A^1)=Y \times \A^1$ and the embedding is the trivial imbedding $X \times \A^1 \hookrightarrow Y \times \A^1$.
\item Over $\infty$, the divisor $M_\infty=\varrho^{-1}(\infty)$ is the sum of two effective Cartier divisors $M_\infty=P(C\oplus 1) + \tilde{Y}$, where $\tilde{Y}$ is the blow-up of $Y$ along $X$.)
\end{enumerate}
The embedding of $X=X \times \{\infty\}$ in $M_\infty$ is the zero section imbedding of $X$ in $C$, followed by the canonical open imbedding of $C$ in $P(C \oplus 1)$. 


If $X$ is a closed subscheme of $Y$, the blow-up of $Y$ along $X$, denoted $\Bl_X Y$ is the projective cone over $Y$ of the sheaf of $\co_Y$-algebras $\oplus \ci^n$, where $\ci$ is the ideal sheaf of $X$ in $Y$.
	\[
	\Bl_X Y = \Proj\left(\bigoplus_{n \geq 0} \ci^n\right)
	\]
Let $\tilde{Y}=\Bl_X Y$ and let $\pi$ denote the projection from $\tilde{Y}$ to $Y$. The canonical invertible sheaf (line bundle) $\co(1)$ on the projective cone $\tilde{Y}$ is the ideal sheaf of $\pi^{-1}(X)$, which is therefore a Cartier divisor on $\tilde{Y}$, called the exceptional divisor. 


When taking Proj of a graded ring or sheaf of graded $\co$-algebras, there is a correspondence between graded modules and sheaves. Under that the $\co(k)$ correspond to shift of the ring itself. The ideal of $\pi^{-1}(X)$ is $\pi^{-1}(\ci)$
	\[
	\ci \cdot \bigoplus_{n \geq 0} \ci^n
	\]
Let $E=\pi^{-1}(X)$. By construction $E$ is the projective cone of $\oplus \ci^n \otimes_{\co(Y)} \co_X=\oplus \ci^n/\ci^{n+1}$ so $E=P(C_X Y)$ the projective normal cone to $X$ in $Y$. $M$ an $A$-module, $I$ an ideal in $A$
	\[
	M \otimes_A A/I \cong M/IM
	\]


In general, $\pi$ induces an isomorphism from $\tilde{Y} \setminus E$ to $Y \setminus X$. On $Y \setminus X$, $\ci \cong \co_Y$. Compare with the blow-up a point in $\A^2$ or $\A^n$. $\pi: \tilde{\A}^n \to \A^n$, $\pi^{-1}(0)=\P^{n-1}$ corresponded to lines through 0 in $\A^n$. Lines through 0 in $\A^n$ is the normal bundle to 0 in $\A^n$. 
	\[
	\begin{tikzcd}
	Z(xy-t) \arrow[draw=none]{r}[sloped,auto=false]{\subseteq} \arrow{dr} & \A^2_{x,y} \times \A^1_t \arrow{d}{p_2} \\
	& \A^1_t
	\end{tikzcd}
	\]
For $t \neq 0$, $p_2^{-1}(t)=\A^2$ inside that $X \cap \A^2$, $xy=t$, the hyperbola. At $t=0$, $xy=0$ and we obtain the axes. You can think of this as a deformation of $xy=0$ or the other way around. For $t \neq 0$, all the curves are isomorphic: $xy=t$, $x=wt, v=y$, then $wtv=t$ and $wv=1$. In some sense, nothing happens until $t=0$ when it ``snaps'' and you get $xy=0$. 


Let $X$ be a closed subscheme of a scheme $Y$ and let $C=C_XY$ be the normal cone to $X$ in $Y$. We will construct a scheme $M=M_XY$ together with a closed imbedding of $X \times \P^1$ in $M$ and a flat morphism
	\[
	\begin{tikzcd}
	X \times \P^1 \arrow[hook]{r} \arrow[swap]{dr}{p^r} & M \arrow{d}{\varrho} \\
	& \P^1
	\end{tikzcd}
	\]
commutes and such that 
\begin{enumerate}[(i)]
\item over $\P^1 \setminus \{\infty\}=\A^1$, $\varrho^{-1}(\A^1)=Y \times \A^1$ and the embedding is the trivial embedding: $X \times \A^1 \hookrightarrow Y \times \A^1$.
\item Over $\{\infty\}$, the divisor $M_\infty=\varrho^{-1}(\infty)$ is the sum of two effective Cartier divisors: 
	\[
	P(C \oplus 1) + \tilde{Y}
	\]
where $\tilde{Y}$ is the blow-up of $Y$ along $X$. The embedding of $X=X \times \{\infty\}$ in $M_\infty$ is the zero section imbedding of $X$ in $C$ followed by one canonical open imbedding of $C$ in $P(C \oplus 1)$. These divisors $P(C \oplus 1)$ and $\tilde{Y}$ intersect in the scheme $P(C)$ which is imbedded as the hyperplane at infinity in $P(C \oplus 1)$ and as the exceptional divisor in $\tilde{Y}$. In particular, the image of $X$ in $M_\infty$ is disjoint from $\tilde{Y}$.
\end{enumerate}

Letting $M^\circ=M^\circ_X Y$ be the complement of $\tilde{Y}$ in $M$, one has a family of imbeddings of $X$:
	\[
	\begin{tikzcd}
	X \times \P^1 \arrow[hook]{r} \arrow[swap]{dr}{p^r} & M^\circ \arrow{d}{\varrho} \\
	& \P^1
	\end{tikzcd}
	\]
which deforms the given imbedding of $X$ in $Y$ to the zero-section imbedding of $X$ in $C_XY$.


We saw a little while ago that we could always interest with the zero-section of a vector bundle. If $X$ sits nicely enough in $Y$ so that $C_XY$ is a bundle not just a cone, we can do intersections. You are just moving things, but not on all of $Y$ just inside $C_XY$ and $C_XY$ sits over $X$ so you really have not moved at all. 


To construct this deformation, let $M$ be the blow-up of $Y \times \P^1$ along the subscheme $X \times \{\infty\}$. Since the normal cone to $X \times \{\infty\}$ in $Y \times \P^1$ is $C \oplus 1$, the exceptional divisor in this blow-up is $P(C \oplus 1)$. $X=X \times \{\infty\} \hookrightarrow X \times \P^1 \hookrightarrow Y \times \P^1$, the blow-up of $X \times \P^1$ along $X \times \{\infty\}$ is imbedded as a closed subscheme of $M$. In particular, if $X \subset Y \subset Z$ are closed imbeddings, there is a canonical imbedding of $\Bl_X Y$ in $\Bl_X Z$ with the exceptional divisor of $\Bl_X Z$ restricting to the exceptional divisor of $\Bl_XY$.


\subsection{Regular Imbeddings and Local Complete Intersections}


\begin{dfn}[Regular Sequence]
A sequence of elements $a_1,\ldots,a_d$ of a ring $A$ is called a regular sequence if the ideal $I$ generated by $a_1,\ldots,a_d$ is a proper ideal of $A$ and the image of $a_i$ in $A/(a_1,a_2,\ldots,a_{i-1})$ is a non-zerodivisor. 
\end{dfn}

For any sequence, regular or not, there is a canonical epimorphism of graded rings 
	\[
	\alpha: A/I[x_1,\ldots,x_d] \ma{} \bigoplus_{n \geq 0} I^n/I^{n+1}
	\]
which takes $x_i$ to the image of $a_i$ in $I/I^2$. If $a_1$ is a non-zerodivisor in $A$, then $A[a_2/a_1,\ldots,a_d/a_1]$ is a subring of the total ring of fractions of $A$ and there is a canonical epimorphism of rings
	\[
	\beta: A[T_2,\ldots,T_d]/I \ma{} A[a_2/a_1,\ldots,a_d/a_1]
	\]
which takes $T_i$ to $a_i/a_1$, where $J$ is the ideal generated by $L_2.\ldots,L_d$ with $L_i=a_1T_i-a_i$. 

\begin{lem}
If $a_1,\ldots,a_d$ is a regular sequence, then $\alpha$ and $\beta$ are isomorphisms and the sequence $L_2,\ldots,L_d$ is a regular sequence in $A[T_2,\ldots,T_d]$.
\end{lem}


\begin{dfn}[Regular Imbedding]
A closed imbedding $i: X \to Y$ of schemes is a regular imbedding of codimension $d$ if every point in $X$ has an affine neighborhood $U$ in $Y$, such that if $A$ is the coordinate ring of $U$, $I$ is the ideal of $A$ defining $X$, then $I$ is generated by a regular sequence of length $d$. If $D$ is an effective Cartier divisor then the inclusion $i: D \to Y$ is a regular imbedding of codimension 1. 
\end{dfn}

For example the three coordinate axes in $\A^3$: $(xy,xz,yz)$ is a non-regular imbedding. If $\ci$ is the ideal sheaf of $X$ in $Y$, it follows that the conormal sheaf $\ci/\ci^2$ is a locally free sheaf on $X$ of rank $d$. The normal bundle to $X$ in $Y$, denoted $N_XY$, is the vector bundle on $X$ whose sheaf of sections is dual to $\ci/\ci^2$. The normal bundle $N_XY$ is canonically isomorphic to the normal cone $C_XY$. Indeed by a previous lemma, the canonical map from $\Sym(\ci/\ci^2)$ to $\oplus \ci^n/\ci^{n+1}$ is an isomorphism. 


\begin{dfn}[Smooth Morphism]
A morphism $f: X \to Y$ of schemes is of finite type over $k$ (a field) is smooth of relative dimension $N$ if:
\begin{enumerate}[(i)]
\item $f$ is flat
\item if $X' \subseteq X$ and $Y' \subseteq Y$ are irreducible components such that $f(X') \subseteq Y'$, then $\dim X'=\dim Y'+n$.
\item for each point $x \in X$ (closed or not)
	\[
	\dim_{k(x)} \left(\Omega_{X/Y} \otimes k(x) \right)=n
	\]
where $k(x)=\co_{x,X}/m_{x,X}$.
\end{enumerate}
\end{dfn}


\begin{thmm}
Let $f: X \to Y$ be a morphism of schemes of finite type over $k$ (a field). Then $f$ is smooth of relative dimension $n$ if and only if 
\begin{enumerate}[(i)]
\item $f$ is flat
\item for each point $y \in Y$, let $X_{\overline{Y}}=X_Y \otimes_{k(x)} \overline{k(y)}$, where $\overline{k(x)}$ is the algebraic closure of $k(y)$.
\end{enumerate}
Then $X_{\overline{Y}}$ is equidimensional of dimension $n$ and regular. [We say ``the fibers of $f$ are geometrically regular of dimension $n$.'']
\end{thmm}


\begin{prop}
Let $f: X \to Y$ be a morphism of nonsingular varieties over an algebraically closed field $k$. Let $n=\dim X - \dim Y$. Then the following conditions are equivalent:
\begin{enumerate}[(i)]
\item $f$ is smooth of relative dimension $n$
\item $\Omega_{X/Y}$ is locally free of rank $n$ on $X$
\item for every closed point $x \in X$, the induced map on the Zariski tangent spaces $T_f: T_X \to T_Y$ is surjective.
\end{enumerate}
\end{prop}

\begin{dfn}[\'{E}tale]
A morphism $f: X \to Y$ of schemes of finite type over $k$ is \et if it is smooth of relative dimension 0. It is unramified if for every $x \in X$, letting $y=f(x)$, we have $\fm_y \co_X=\fm_X$, and $k(x)$ is a separable extension of $k(y)$. One can show the following are equivalent:
\begin{enumerate}[(i)]
\item $f$ is \et.
\item $f$ is flat and $\Omega_{X/Y}=0$
\item $f$ is flat and unramified.
\end{enumerate}
\end{dfn}

\Et is the algebraic version of a covering space in Topology. 

\begin{dfn}[Local Complete Intersection]
A morphism $f: X \to Y$ will be called a local complete intersection (l.c.i.) morphism of codimension $d$ if $f$ admits a factorization into a closed (regular) imbedding of some codimension $e$, followed by a smooth morphism of relative dimension $d+e$.
\end{dfn}

If $Y$ has pure dimension $k$ then $C_X Y$ also has pure dimension $k$, for any closed subscheme $Y \subset Y$. Let $X$ be a closed subscheme of a scheme $Y$, let $C=C_X Y$ be the normal cone. Define the specialization homomorphisms $\sigma: Z_k Y \to Z_kC$ by the formula $\sigma[V]=[C_{V \cap X} V]$ for any $k$-dimensional subvariety $V$ of $Y$ and extending linearly to all $k$-cycles. Note that $C_{V \cap X} V$ is a purely $k$-dimensional scheme so it has a fundamental cycle $[C_{V \cap X}V]$. One needs that $C_{V \cap X}V$ is a subscheme of $C_X Y=C$:
	\[
	\begin{tikzcd}
	X \arrow[draw=none]{r}[sloped,auto=false]{\subseteq} \arrow[draw=none]{d}[sloped,auto=false]{\supseteq} & Y \arrow[draw=none]{d}[sloped,auto=false]{\supseteq} \\
	V \cap X \arrow[draw=none]{r}[sloped,auto=false]{\subseteq} & V
	\end{tikzcd}
	\]	
and if $\ci$ is the ideal sheaf of $X$ in $Y$ and $\mathcal{J}$ is the ideal sheaf of $V\cap X$ in $V$, $\ci \mathrel{\rotatebox{0}{$\twoheadrightarrow$}} \mathcal{J}$ so that $C_{V \cap X} V \hookrightarrow C_X Y$.

\begin{prop}
If a cycle $\alpha \in Z_kY$ is rationally equivalent to zero on $Y$, then $\sigma(\alpha)$ is rationally equivalent to zero on $C$. Therefore, $\sigma$ passes to rational equivalence, defining specialization homomorphisms 
	\[
	\sigma: A_k Y \to A_k C
	\]
\end{prop}




\section{Intersection Products}

Let $i: X \to Y$ be a (closed) regular imbedding of codimension $d$ and denote the normal bundle by $N_X Y$. Let $V$ be a purely $k$-dimensional scheme and let $f: V \to Y$ be a morphism. [You might first guess $V$ should be a $k$-dimensional subvariety of $Y$. This is a special case, let $f=i: V \hookrightarrow Y$ be the inclusion.] Denote the inverse image scheme $f^{-1}(X)$ by $W$ and form the fiber square
	\[
	\begin{tikzcd}
	W \arrow{r}{i} \arrow{d}{g} & V \arrow{d}{f} \\
	X \arrow{r}{i} & Y
	\end{tikzcd}
	\]
Fiber square means $W= X \times_Y V$. That is, $f^{-1}(X)$ is $W=X \times_Y V$. Let $N=g^*N_XY$, a bundle of rank $d$ on $W$, and let $\pi: N \to W$ be the projection. Since the ideal sheaf $\ci$ of $X$ in $Y$ generates the ideal sheaf $\mathcal{J}$ of $W$ in $V$, $\{W=f^{-1}(X)\}$, there is a surjection 
	\[
	\begin{tikzcd}
	\bigoplus_n f^*\left(\ci^n/\ci^{n+1}\right) \arrow[two heads]{r} & \bigoplus_n \left(\mathcal{J}^n/\mathcal{J}^{n+1}\right)
	\end{tikzcd}
	\]
This determines a closed imbedding of the normal cone $C=C_WV$ as a subcone of the vector bundle $N$
	\[
	\begin{tikzcd}
	C \arrow{dr} \arrow[hook]{r} & N \arrow{d}{\pi} \\ & W
	\end{tikzcd}
	\]
Since $C$ is purely $k$-dimensional, it determines a $k$-cycle $[C]$ on $N$. Let $s$ be the zero section of the bundle $N$. Define the intersection product of $V$ by $X$ on $Y$, denoted by $X \cdot V$ (or $X \cdot_Y V$ or $i^![V]$) to be the class on $W$ obtained by intersection $[C]$ by the zero section of $N$. 


Now $X \cdot V = s^*[C]$ in $A_{k-d}W$, where $s^*: A_k N \to A_{k-d} W$ is the Gysin homomorphism. Equivalently, $X \cdot V$ is the unique class in $A_{k-d}W$ such that $\pi^*(X \cdot V)=[C]$ in $A_kN$.  It was proven that $\pi^*$ was an isomorphism from $A_kW$ to $A_{k+d}N$. $s^*$ was defined to be $(\pi^*)^{-1}$. 


Think about this some more in the case that $V$ is a $k$-dimensional subvariety of $Y$.
	\[
	\begin{tikzcd}
	W \arrow{r}{i} \arrow{d}{g} & V \arrow{d}{f} \\ X \arrow{r}{i} & Y
	\end{tikzcd}
	\]
where $f$ is the inclusion of $V$ in $Y$ and $i$ is the inclusion of $X$ in $Y$. Now $X,V$ are both closed in $Y$ and closed imbeddings are proper. Proper morphisms are preserved under fiber products. So $j,g$ are proper and $X \cdot V \in A_{k-d}W$. Pushing down to $Y$ via the proper push forward $i_*g_*$ or $f_*j_*$, we obtain something in $A_{k-d}Y$. But $V$ has dimension $k$ and $X$ has codimension $d$. The image of $ig$ or $fj$ is $X \cap V$. One actually obtains a class in $A_{k-d}(X \cap V)$. 

\begin{prop} \hfill
\begin{enumerate}[(i)]
\item With the above notation, 
	\[
	X \cdot V = \{c(N) \cap s(W,V) \}_{k-d}
	\]
the $(k-d)$th dimensional piece. 
\item If $\xi$ is the universal quotient bundle of rank $d$ on $P(N \oplus 1)$ and $q$ is the projection from $P(N \oplus 1)$ to $W$, then 
	\[
	X \cdot V= q_*\left(c_d(\xi) \cap [P(C \oplus 1)]\right)
	\]
What is the universal quotient bundle?
	\[
	\begin{tikzcd}
	P(N \oplus 1) \arrow{dr}{q} & N \oplus 1 \arrow{d} \\
	& W
	\end{tikzcd}
	\]
We have $0 \ma{} \co(-1) \ma{} q^*(N \oplus 1)$ the tautological bundle so that 
	\[
	0 \ma{} \underbrace{\co(-1)}_{\text{rank }1} \ma{} \underbrace{q^*(N \oplus 1)}_{\text{rank } d+1}  \ma{} \underbrace{\xi}_{\text{rank }d} \ma{} 0
	\]
\item If $d=1$, i.e. $X$ is an effective Cartier divisor on $Y$, $V$ is a variety, and $f$ is a closed imbedding, then $X \cdot V$ is the intersection class constructed previously constructed. 
\end{enumerate}
\end{prop}


% "Final Exam' Tuesday December 12 Carnegie 100 12:45 - 2:45

Our final topics will be the Grothendieck-Riemann-Roch Theorem, Porteous's Formula, Bezout's Theorem, and some about excess and residual intersection. For the last we give a question: what can you say when the dimension of the intersection is bigger than expected? For example, take in $\P^3$ three hypersurfaces of degrees $d$, $e$, and $f$. We expect they will intersect in $def$ points. What if they intersect in a curve and perhaps some points?

\begin{ex}
Take the twisted cubic curve $\P^1 \to \P^3$ via $[x_0,x_1] \mapsto [x_0^3,x_0^2x_1,x_0x_1^2,x_1^3]:=[z_0,z_1,z_2,z_3]$.  This lies on 3 obvious quadratics: $A: z_1z_2-z_0z_3=0$, $B: z_1^2-z_0z_2=0$, $C: z_2^2-z_1z_3=0$. We know $A \cap B \cap C$ is the twisted cubic curve. They should intersect in 8 points. `The twisted cubic curve absorbed all 8 points.' We have $A \cap B$ being the twisted cubic plus the line $z_0=z_1=0$. The line and cubic meet at $(0,0,0,1)$ and are tangent there. Evaluating $C$ at $z_0=z_1=0$ gives $z_2=0$. This happens on the patch $z_3=1$. So $x_1=1$ and the twisted cubic is $t \mapsto (t^3,t^2,t)$. Restricting $z_0=z_1=0$ to this curve is the ideal $(t^3,t^2)$---multiplicity 2. Define $D: z_0z_2+z_1z_3+z_0^2=0$. This contains the line $z_0=z_1=0$ but not the twisted cubic. How many of the 8 points does the line absorb? Work on the patch $z_3=1$. If $z_3=0$, $[x_0,x_1] \mapsto [x_0^3,x_0x_1,x_0x_1^2,x_1^3]$ and if $x_1=0$, then we have the point $[1,0,0,0]$ but this is not in $D$. Consider again $t \mapsto (t^3,t^2,t):=(z_0,z_1,z_2)$. We have $ t^4+t^2+t^6=t^2(t^4+t^2+1)=0$ so that $t^2=0$ and these two points are on the line but the four points on $t^4+t^2+1=0$ are not on the line. The line only absorbed 4 of the 8 points. Four points were `left over'. 
\end{ex}


\begin{ex}
Let $l_1,\ldots,l_4$ be 4 general lines in $\P^2$ and let $P$ be a point of $\P^2$ not on any $l_i$. Furthermore, assume $P$ is not on any line joining pairwise intersections of two lines $l_i$. How many conics through $P$ tangent to all of $l_1,\ldots,l_4$? Consider the space of all conics is isomorphic to $\P^5$: there are 6 monomials, $ax_0^2+bx_1^2+cx_2^2+dx_0x_1+ex_0x_2+fx_1x_2=0$. For a fixed line, conics tangent to that line are a hypersurface: $H_1,\ldots,H_4$. Conics through $P$ are a hyperplane, say $H_p$. Bezout's Theorem gives $\deg H_p \cdot \prod_{i=1}^4 \deg(H_i)$. This number is wrong. Any double line through $P$ is a conic tangent to all the $l_i$. Then there is a positive dimensional component in the intersection. 
\end{ex}



\subsection{Refined Gysin Homomorphism}



Let $i: X \to Y$ be a regular imbedding of codimension $d$ and let $f: Y' \to Y$ be a morphism. Form the fiber square
	\[
	\begin{tikzcd}
	X' \arrow[swap]{d}{g} \arrow{r}{j} & Y' \arrow{d}{f} \\
	X \arrow{r}{i} & Y
	\end{tikzcd}
	\]
Define homomorphisms $i^!: Z_kY' \to A_{k-d} X'$ by the formula $i^!\left(\sum n_i[V_i]\right)= \sum n_i X \cdot V_i$, where $X \cdot V_i$ is the intersection product constructed in the previous section. Now $X \cdot V_i$ is a cycle class on $X' \cap V_i$; it really means $X' \cap j^{-1}(V_i)$. Now $X' \cap j^{-1}(V_i)$ is closed in $X'$. The proper push forward via the inclusion map gives class in $A_{k-d} X'$. This passes to a rational equivalence. Now $i^!: A_k Y' \to A_{k-d} X'$ are called the refined Gysin homomorphisms. If $Y'+Y$, $f=1_Y$, denote these maps $i^*: A_kY \to A_{k-d}X$ and call them Gysin homomorphisms. 

\begin{thmm}
Consider a fiber diagram
	\[
	\begin{tikzcd}
	X'' \arrow[swap]{d}{q} \arrow{r}{i''} & Y'' \arrow{d}{p} \\
	X' \arrow[swap]{d}{g} \arrow{r}{i'} & Y' \arrow{d}{f} \\
	X \arrow{r}{i} & Y
	\end{tikzcd}
	\]
with $i$ a regular imbedding of codimension $d$. 
\begin{enumerate}[(a)]
\item (Push-forward) If $p$ is proper and $\alpha \in A_k Y''$ then $i^! p_*(\alpha)=q_*(i^!\alpha)$ in $A_{k-d}X'$. 
\item (Pull-back) If $p$ is flat of relative dimension $n$ and $\alpha \in A_kY'$ then $i^! p^*(\alpha)=q^*i^!(\alpha)$ in $A_{k+n-d} X''$. 
\item (Compatibility) If $i'$ is also a regular imbedding of codimension $d$ and $\alpha \in A_k Y''$ then $i^! \alpha= i'^! \alpha$ in $A_{k-d}X''$.
\end{enumerate}
\end{thmm}



\subsection{Bezout's Theorem}



Let $X$ be a set and $Y_1,\ldots,Y_t$ be subsets of $X$. Consider $Y_1 \times \cdots \times Y_t \subseteq X \times \cdots \times X$. We have the diagonal map $\delta: X \to X \times \cdots \times X$ given by $\delta(x)=(x,x,\ldots,x)$. Then $\delta^{-1}(Y_1\times \cdots \times Y_t)=Y_1 \cap \cdots \cap Y_t$. For a scheme, we also have a diagonal morphism $\delta: X \to X \times_{\Spec} \cdots \times_{\Spec} X$.

\begin{prop}
Assume that $X$ is separated, of finite type over $\spec k$, and is nonsingular. Then $X$ is a regular embedding of codimension $(t-1)\dim X$.
\end{prop}

\pf This is a local question. We can look in the local rings or even their completions. Say $X$ has dimension $n$. At a point, the local ring looks like $k\llbracket x_{1,1},\ldots,x_{1,n},x_{2,1},\ldots,x_{2,n},\ldots,x_{t,1},\ldots,x_{t,n} \rrbracket$. Assuming the point was on the diagonal, the local equations for $\delta(x)$ are $x_{1,i}=x_{2,i}=\cdots=x_{t,i}$, $i=1,\ldots,n$. The ideal generated by $x_{1,i}-x_{2,i},x_{1,i}-x_{3,i},\ldots,x_{1,i}-x_{n,i}$, $i=1,\ldots,t$. This is a regular sequence because these are independent $k$-linear combinations of $x_{i,j}$. \qed \\

\begin{ex}
Let $\delta$ be the diagonal imbedding of $\P^n$ in $\P^n \times \P^n$ ($r$-factors). Let $[k]$ denote the generator of $A_k \P^n$ given by a $k$-plane in $\P^n$. $A_k\P^n=\Z$, $k=0,\ldots,n$. We have generator $[L_k]$, where $L_k$ is linear of dimension $k$. $H$ is a hyperplane at infinity $\A^n=\P^n \setminus H$. 
	\[
	A_k(H) \ma{} A_k(\P^n) \ma{} A_k(\A^n) \ma{} 0
	\]
Then the Gysin homomorpihsm is determined by the formula
	\[
	\delta^*([k_1] \times \cdots \times [k_r])=[l]
	\]
where $l=k_1+\cdots+k_r-(r-1)n$. Now $\delta^*(*)$ has to be in $A_l\P^n$ generated by $[l]$. It must be some multiple of $[l]$. We have $A_k(\P^n \times \cdots \times \P^n$ is given by
	\[
	\bigoplus_{i_1+\cdots+i_r=k} A_{i_1} \P^n \otimes_\Z \cdots \otimes_\Z A_{i_r} \P^n
	\]
This can be proven using a more complicated version of what you did for $\P^n$ or realize $\P^n \times \cdots \times \P^n$ ($r$ times) is a trivial projective bundle of rank 1 over $\P^n \times \cdots \times \P^n$ ($r-1$ times). Use results about $A_k$ of projective bundles. 
\end{ex}

\begin{rem}
The analogous result is often not true for arbitrary schemes, even if they are projective, nonsingular, varieties. Sometimes you get more. [Take the linear spaces in general position apply Theorem 6.2(c) in Fulton.] As an alternative proof, return to the construction of $X \cdot V$
	\[
	\begin{tikzcd}
	W \arrow[swap]{d}{g} \arrow{r}{j} & V\arrow{d}{f} \\
	X \arrow{r}{i} & Y
	\end{tikzcd}
	\] 
Using a previous proposition with the above notation, $X \cdot V=\{c(N) \cap s(W,V)\}_{k-d}$. Using another proposition, if $E$ is a vector bundle on $X$, then $s(E)=c(E)^{-1} \cap [X]$, where $c(E)=1+c_1(E)+\cdots+c_r(E)$ is the total Chern class of $E$, $r=\rank E$. Let $X$ be a closed subscheme of a scheme $Y$. Let $C=C_XY$ be the normal cone to $X$ in $Y$.
	\[
	s(X,Y)=s(C_XY) \in A_*X.
	\]
In case $X$ is regularly imbedded in $Y$, so the normal cone is a vector bundle, it follows from a previous proposition that $s(X,Y)$ is the cap product of the total inverse Chern class of the normal bundle with $[X]$. Applying this to our situation becomes
	\[
	\begin{tikzcd}
	l \arrow{r} \arrow[draw=none]{d}[sloped,auto=false]{\subseteq} & k_1 \times \cdots \times k_r \arrow[draw=none]{d}[sloped,auto=false]{\subseteq} \\
	\P^n \arrow{r}{\delta} & \P^n \times \cdots \times \P^n
	\end{tikzcd}
	\]
If you took the linear spaces $k_1,\ldots,k_r$ in general position, their intersection will have dimension $l$. $k_i$ has codimension $n-k$ in $\P^n$. If $l \geq 0$, the sum of codimensions at most $n$ locally, $k_1$ first $n-k_1$ coordinates is 0, $k_2$ second $n-k_2$ coordinates is 0, and so forth until $k_r$ the next $n-k_r$ coordinates is 0. Then no two of the $k_i$ ever involve setting the same coordinate to 0. None of the $(r-1)n$ equations for $\delta(X) \subset \P^n \times \cdots\times \P^n$ get wasted when you restrict to $k_1 \times \cdots \times k_r$. They will still be a regular sequence inside $k_1 \times \cdots \times k_r$. $l$ is regularly imbedded in $k_1 \times \cdots \times k_r$ so you can compute the Segre class via Chern classes of normal bundles. 
	\[
	X \cdot V= \{ c(N) \cap s(l,k_1 \times \cdots \times k_r)\}_{k-d=l}
	\]
And $c(N)=1+c_1(N)+c_2(N) \cap \left(c(E)^{-1} \cap [l]\right)$ is $1+\text{ higher Chern }+ \text{codim }1$. 
\end{rem}


For $V_1,\ldots,V_r$ closed subschemes of $\P^n$ with $V_i$ of pure dimension $K_i$, it follows that $\delta^!=\delta^*[V_1\times \cdots \times V_r]$ is a cycle class in $A_i(\cap V_i)$ whose degree is the product of the degrees of the $V_i$. 
	\[
	\deg \delta^*[V_1\times \cdots \times V_r]=\prod_i \deg [V_i]
	\]
$[V_i]=\deg V_i[k_i]$, $\delta^*[[k_1] \times \cdots \times [k_r]]=[l]$. This is a generalization of Bezout.




\subsection{Intersection Multiplicities}




Now we look at intersection multiplicities. This is not as simple as before.

\begin{prop}
Suppose you expect $V_1 \cap \cdots \cap V_r$ to be zero dimensional, $k$ algebraically closed, and locally near $P$ the ideals of $V_1,\ldots,V_r$ generate the ideal of $P$ (thus $P$ is an isolated point of $V_1 \cap \cdots \cap V_R$). Then $P$ is counted with multiplicity one.  
\end{prop}

\pf This is the same as the proof that the coefficient of $[l]$ was one. \qed \\

Now we return to the more general setup
	\[
	\begin{tikzcd}
	W \arrow{r}{j} \arrow[swap]{d}{g} & V \arrow{d}{f} \\
	X \arrow{r}{i} & Y
	\end{tikzcd}
	\]

\begin{dfn}
An irreducible component $Z$ of $W=f^{-1}(X)$ is a proper component of the intersection of $V$ by $X$ if $\dim Z=k-d$. ($k=\dim V$ and $d=\codim X$). The intersection multiplicity of $Z$ in $X$ is denoted 
	\[
	i(Z, X \cdot V; Y)
	\]
or simply $i(Z, X \cdot V)$ or $i(Z)$ is the coefficient of $Z$ in the class $X \cdot V$ in $A_{k-d}(W)$. 
\end{dfn}


\begin{prop}
Assume $Z$ is a proper component 
\begin{enumerate}[(a)]
\item $1 \leq i(Z,X \cdot V; Y) \leq l(A/J)$
\item If $J$ is generated by a regular sequence of length $d$, then $i(Z,X \cdot V; Y)=l(A/J)$.
\end{enumerate}
 If $A$ is Cohen-Macaulay (e.g. regular) the local equations for $X$ in $Y$ give a regular sequence generating $J$ and equality in (b) holds.
\end{prop}

Let $A=\co_Z V$ and $J$ be the ideal in $A$ generated by the ideal of $X$ in $Y$, $\fm$ the maximal ideal of $A$.

\begin{prop}
Assume $Z$ is a proper component and $V$ is a variety. Then the following are equivalent:
\begin{enumerate}[(i)]
\item $i(Z, X\cdot V;Y)=1$
\item $A$ is a regular local ring and $J=\fm$. 
\end{enumerate}
\end{prop}


\section{Excess and Residual Intersections}

Let $Y$ be a scheme, $X_i \hookrightarrow Y$ regularly imbedded subschemes, $1 \leq i \leq r$, and $V$ a $k$-dimensional subvariety of $Y$. The intersection product $X_1 \cdot \ldots \cdot X_r \cdot V$ is a class in $A_m(\cap X_i \cap V)$, $m=\dim V - \sum_{i=1}^r \codim(X_i,Y)$. It is constructed by the procedure of Section 6.1 of Fulton applied to the diagram
	\[
	\begin{tikzcd}
	\cap X_i \cap V \arrow[hook]{d} \arrow[hook]{r} & V \arrow[hook]{d}{\delta} \\
	X_1 \times \cdots \times X_r \arrow[hook]{r} & Y \times \cdots \times Y
	\end{tikzcd}
	\]
If $Z$ is a connected component of $\cap X_i \cap V$, we write $(X_1 \cdot \ldots \cdot X_r \cdot V)^Z \in A_mZ$ for the part of $X_1 \cdot \ldots \cdot X_r \cdot V$ supported on $Z$ and call it the equivalence of $Z$ for the intersection $X_1 \cdot \ldots \cdot X_r \cdot V$. Remember $Z$ is a connected component not an irreducible component. Consider class $\alpha \in A_m(\cap X_i \cap V)$. To what extent can you say where it lives? Say $Z_1$ and $Z_2$ were irreducible components of $\cap X_i \cap V$ but $Z_1 \cap Z_2$ had dimension at least $m$. You can not say what part of $\alpha$ lives on each $Z_i$. If $Z_1 \cap Z_2=\emptyset$, then there is no way to move from $Z_1$ to $Z_2$.

\begin{prop}
Let $N_i$ be the restriction of $N_{X_i}Y$ to $Z$. Then $(X_1 \cdot \ldots \cdot X_r \cdot V)^Z=\{\prod_{i=1}^r c(N_i) \cap s(Z_iV)\}_m$. If $Z$ is regularly imbedded in $V$ with normal bundle $N_Z V$, then $(X_1 \cdot \ldots \cdot V)^Z=\{\prod_{i=1}^r c(N_i) \cdot c(N_ZV)^{-1} \cap [Z]\}_m$. If $V$ and $Z$ are non-singular, then $(X_1 \cdot \ldots \cdot X_r V)^Z=\{\prod_{i=1}^r c(N_i) c(T_{V/Z} )c(T_Z) \cap [Z] \}_m$.
\end{prop}

Note that for regular imbeddings the normal cone is the normal bundle. 

\begin{ex}
On $\P^n$, we have an exact sequence
	\[
	0 \ma{} \co_{\P^n} \ma{} \co_{\P^n}(1)^{n+1} \ma{} \mathcal{T}_{\P^n} \ma{} 0
	\]
Exponent $n+1$ means direct sum $n+1$ times, $\mathcal{T}_{\P^n}$ is the tangent bundle to $\P^n$. $c(\mathcal{T}_{\P^n})=c(\co_{\P^n}(1)^{n+1})=\prod_{i=1}^{n+1}c(\co_{\P^n}(1))$. In particular, if $i: X \to Y$ imbeds $X$ as a Cartier divisor on $Y$, then $C_XY = N_X Y=i^*\co_Y(X)$. Cartier divisors correspond to a line bundle $\co_Y(X)$ the line bundle on $Y$ induced by $X$. $i^*$ is the restriction to $X$.
\end{ex}

\begin{ex}
The twisted cubic as the intersection of three quadrics in $\P^3$. Take $X_i=Q_i$ three quadrics. Let $Y=\P^3$ and $V=\P^3$. We have $\cap X_i= \cap X_i \cap V= Z$, the twisted cubic curve, which is $\P^1$. Call the twisted cubic $T$. If $N_i$ is the normal bundle to $X_i$ in $Y$ restricted to $Z$, $Q_i=X_i$. Now $N_i$ is two hyperplanes in $\P^3$ restricted to $T$ which is a curve of degree 3---6 times a point. Further, $T \cong \P^1$ and $A_0\P^1=\Z$. The first term is $(1+6[P])^3$
	\[
	0 \ma{} \co_{\P^n} \ma{} \co_{\P^n}(1)^{n+1} \ma{} \mathcal{T}_{\P^n} \ma{} 0
	\]
We have $c(\mathcal{T}_{\P^n})=(1+H)^{n+1}$. Now $T$ is the cubic and $H \cap T$ is three 3 points ($n=3$, $\P^3$). The second term is $\frac{1}{(1+3[P])^4}$ while the third term is $(1+[P])^2$. Putting this together, 
	\[
	(1+6[P])^3 \cdot \dfrac{1}{(1+3[P])^4} \cdot (1+[P])^2= (1+18[P]) \cdot \dfrac{1}{1+12[P]} \cdot (1+2[P])
	\]
Using geometric series, we have
	\[
	(1+18[P])(1-12[P])(1+2[P])=18[P]-12[P]+2[P]=8[P]
	\]
so that the twisted cubic has absorbed 8 points so there are none left over. If there are three quadrics intersecting in a line, how does the calculation change?
	\[
	(1+2[P])^3 \cdot \dfrac{1}{(1+[P])^4} \cdot (1+[P])^2=6[P] - 4[P] + 2[P] = 4[P]
	\]
The line absorbs 4 points so there should be 4 left over. 
\end{ex}



\subsection{Grothendieck-Riemann-Roch Theorem}

First, we will discuss the classical Riemann-Roch Theorem. Let $C$ be an irreducible, nonsingular, projective curve over an algebraically closed field $k$. [The nonsingular is so Cartier divisors if and only if they are Weil divisors.] A divisor on $C$ is $D \in Z_0C=\sum n_p P$, $n_p \in \Z$, $P \in C$ and at most finitely many of the $n_p$ are nonzero. $Z_0C$ is called $\Div C$. The degree of $D$ is $\sum n_p$. We say $D$ is effective if and only if $n_p \geq 0$. For a rational function $f \in R(C)=k(C)$, the divisor of $f$ is $\div(f)= \sum \ord_P(f) P$. It can be shown that this is a finite sum. If $\div f= \div g$ then $f=cg$ for some $c \in k$. [If $f/g$ is regular on $C$, then it is constant.] Moreover, $\deg \div f=0$. To see this, the canonical map $\pi: C \to \spec k$ is proper. We can do the proper push forward on $\pi_*: A_0 C \to A_0 \spec k$. Now $\div f=0$ is $A_0C$ and $\spec k$ is a point. Further, $\pi_* \sum n_p P=\sum n_p \spec k$ so that $\sum n_p=0$ for $\div f$. 


A rational equivalence is called a linear equivalence $D \sim E$ if and only if $D-E=\div f$. Now $\Pic C=A_0C=Z_0C/\sim$. If $D \sim E$ then $\deg D=\deg E$. 
	\[
	\mathcal{L}(D)=\{f \in k(C) \colon D+ \div f \geq 0\}
	\]
The zeros of $f$ cancel any negatives in $D$. The poles of $f$ less than the positives in $D$. Now $\mathcal{L}(D)$ is a vector space over $k$. Now $l(D)=\dim_k \mathcal{L}(D)$. The Riemann-Roch problem is to figure out what $l(D)$ is. $|D|$ is called the complete linear system of $D$ and is defiend to be the set $\{E \in \div C \colon E \sim D \text{ and } E \geq 0\}$. Now $|D|=\P(\mathcal{L}(D))$. Now $f \in \mathcal{L}(D)$ if and only if $D+\div f \in |D|$ and $f,g \in \mathcal{L}(D)$ gives the same element of $|D|$ if and only if $f=cg$.


What is the connection with geometry? Suppose $C \subseteq \P^n$. Let $H \subseteq \P^n$ be a hyperplane and set $D=H \cap C$ (with multiplicity). For any other $H'$ hyperplane in $\P^n$, $E= H' \cap C$ (with multiplicity). Now if $E \in |D|$, $\frac{H}{H'}$ is a rational function whose divisor is $D-E$. All hyperplanes in $\P^n$ is a $\P^n$. This gives a map $\P^n \hookrightarrow |D|=\P(\mathcal{L}(D))$. The sheaf of differentials $\Omega_{C/k}$ is locally free rank 1. It comes from a divisor, call in $K$ for canonical divisor (only defined up to linear equivalence). Define the genus of $C=l(K)$. This agrees with the topological definition of genus when $k=\C$. If $D\sim E$, then $l(D)=l(E)$.

\begin{thmm}[Riemann-Roch Theorem]
	\[
	\underbrace{l(D) - l(K-D)}_{\text{analytical}}= \underbrace{\deg D - \text{ genus }C+1}_{\text{topological}}
	\]
\end{thmm}


Plugging in $D=K$, we have $\text{genus }C=l(K)-l(0)=\deg K - \text{ genus }C+1$. Now $\mathcal{L}(0)$ are the constants and $\deg k = 2 \text{ genus }C-2$. 


\begin{lem}
If $\deg D<0$, then $l(D)=0$.
\end{lem}

\pf For any rational function $\deg \div f=0$. So $\deg(D+\div f)<0$ so $D+\div f$ is not at least 0. \qed \\


\begin{prop}
If $\deg D > 2g-2$, then $l(D)=\deg D-g+1$.
\end{prop}

\pf We know $\deg K=2g-2$ so $\deg(K-D)<0$. Therefore, $l(K-D)=0$ and substituting this into the Riemann-Roch Theorem gives the result. \qed \\

The only `gray area' is $0 \leq \deg D \leq 2g-2$. We now move to the Grothendieck-Riemann-Roch Theorem.

\begin{dfn}[The Grothendieck Group]
Let $X$ be a noetherian scheme. We define $K(X)$ to be the free group generated by all the (isomorphism classes of) coherent sheaves on $X$ modulo the subgroup generated by all expressions of the form $\mathcal{F} - \mathcal{F}' - \mathcal{F}''$ whenever there is an exact sequence
	\[
	0 \ma{} \mathcal{F}' \ma{} \mathcal{F} \ma{} \mathcal{F}'' \ma{} 0
	\]
of coherent sheaves on $X$. 
\end{dfn}

Strictly speaking all coherent sheaves on $X$ is not a set. The zero sheaf is the identity as
	\[
	0 \ma{} \mathcal{F} \ma{=} \mathcal{F} \ma{} 0 \ma{} 0
	\]
Isomorphic sheaves have the same class:
	\[
	0 \ma{} \mathcal{F}' \ma{\cong} \mathcal{F} \ma{} 0 \ma{} 0
	\]
Let $X$ be a noetherian, integral, separated, regular ($=$nonsingular) scheme. Define $K_1(X)$ just as one defined $K(X)$ but use locally free coherent sheaves instead of all coherent sheaves. Clearly there is a natural group homomorphism $\epsilon: K_1(X) \to K(X)$. One can show $\epsilon$ is an isomorphism (Borel and Serre) as follows:
\begin{enumerate}[(a)]
\item Given a coherent sheaf $\mathcal{F}$, show that it has a locally free resolution $\mathcal{E}_\cdot \ma{} \mathcal{F} \ma{} 0$. Then one can show this has a finite locally free resolution
	\[
	0 \ma{} \mathcal{E}_n \ma{} \cdots \ma{} \mathcal{E}_1 \ma{} \mathcal{E}_0 \ma{} \mathcal{F} \ma{} 0
	\]
\item For each $\mathcal{F}$, choose a finite locally free resolution $\mathcal{E}_\cdot \ma{} \mathcal{F} \ma{} 0$ and let $\delta(\mathcal{F})=\sum (-1)^i \gamma(\mathcal{E}_i)$ in $K_1(X)$. $\gamma(\mathcal{E}_i)$ is the class of $\mathcal{E}_i$ in $K_1$. Show that $\delta(\mathcal{F})$ is independent of the resolution chosen and that it defines a homomorphism of $K(X)$ to $K_1(X)$ and finally that it is inverse to $\mathcal{E}$.
\end{enumerate}


We saw in Fulton Example 3.2.3. that if $0 \ma{} E' \ma{} E \ma{} E'' \ma{} 0$ is an exact sequence of vector bundles then the Chern character satisfies $\ch E = \ch E' + \ch E''$. Thus $\ch$ is well defined on $K_1(X)$ and thus for nice schemes $K(X)$. 


We saw before that if $f: X \to Y$ is any morphism of topological spaces and $\mathcal{F}$ is a sheaf of abelian groups on $X$ then we can define the push forward $f_*\mathcal{F}$ as a sheaf of abelian groups on $Y$: $f_*\mathcal{F}(U)=\mathcal{F}(f^{-1}(U))$. It turns out that $f_*$ is a left exact functor that is not exact. The categories involved have enough injectives so you can define right derived functors $R^if_*\mathcal{F}$. These are called the higher direct images. Some facts:
\begin{enumerate}[(i)]
\item $R^if_*(\mathcal{F})$ is the sheaf associated to the presheaf $V \mapsto H^i(f^{-1}(V), \mathcal{F}\big|_{f^{-1}(V)})$.
\item If $f: X \to Y$ is a proper morphism of noetherian schemes and $\mathcal{F}$ is a coherent sheaf on $X$, then $R^if_*(\mathcal{F})$ is a coherent sheaf on $Y$. 
\end{enumerate}

Let $f: X \to Y$ be a proper morphism of schemes. We define $f_*: K(X) \to K(Y)$ by $f_*[\mathcal{F}]=\sum_{i=0}^\infty (-1)^i [R^if_*(\mathcal{F})]$. It will actually be a finite sum for $i> \dim X$, $H^i(f^{-1}(V), \mathcal{F}\big|_{f^{-1}(V)})=0$.

\begin{thmm}[Grothendieck-Riemann-Roch Theorem]
Let $f: X \to Y$ be a proper morphism of non-singular varieties. Then for all $\alpha \in K(X)$, $\ch(f_*\alpha) \cdot \text{td}(T_Y)=f_*(\ch(\alpha) \cdot \text{td}(T_X))$ in $A(Y)_\Q$.
\end{thmm}


You can think of $A(Y)_\Q$ as $A(Y) \otimes_\Z \Q$. On nonsingular varieties, Fulton was able to define intersection pairings, $n=\dim Y$, $A_{n-i}Y \otimes A_{n-j} Y \ma{} A_{n-i-j}Y$. $A(Y)$ is just $A_*(Y)$ but remembering we can do intersections. Chow classes were things you could intersect with. $c_i(E)$ is the same as $c_i(E) \cap [Y]$. Let's derive the old Riemann-Roch Theorem from this. Let $C$ be a non-singular projective curve over $k$, algebraically closed. The map $f: C \to \spec k$ is proper because $C$ is projective. Let $D$ be a divisor on $C$ and take $\alpha=\co_C(D)$. On the left side fo the theorem we have $\ch(f_*\alpha) \cdot \td(T_Y)$.
	\[
	f_*(\co_C(D))=H^0(C,\co_C(D)) - H^1(C,\co_C(D))
	\]
$\td(T_Y)=1$. Given a vector bundle $E$, we have $\ch(E)=r+\cdot$, where $r=\rank E$ and $\dim_k H^0(C,\co_C(D)) - \dim_k H^1(C,\co_C(D))$. On the right side, $\ch(\co_C(D))=1+D$, $\ch(E)=r+c_1+\cdots$, $\mathcal{L}(D)=H^0(C,\co_C(D))$. We have $\td(E)=1+\frac{1}{2}c_1$, $\td(T_X)=1- \frac{1}{2}K$, and
	\[
	(1+D)(1-\frac{1}{2}K)=1+(D- \frac{1}{2} K)
	\]
Now $l(D)=\dim_k H^0(C,\co_C(D))$. If $X$ is a nonsingular variety over $k$, we define the canonical bundle $\omega_X= \wedge^n \Omega_{X/k}$, where $n=\dim X$. It is an invertible sheaf, line bundle, divisor on $C$. Combining previous corollaries, one obtains

\begin{thmm}[Kodaira-Serre Duality]
Let $X$ be a nonsingular projective variety of dimension $n$ and $\mathcal{L}$ a line bundle on $X$
	\[
	H^i(X,\mathcal{L}) \cong H^{n-i}(X,\omega_X\otimes \mathcal{L}^*)^*.
	\]
\end{thmm}

Then $H^!(C,\co_C(D)) \cong H^0(C,K-D)^*$, where $i=1,n=1$. Also, the dimension is $l(K-D)$. 


There are further generalizations. For singular varieties, there is Baum-Fulton-MacPherson. One wants to `get rid' of the $\otimes \Q$ term and `get rid' of the over $k$ restriction and replace this with $\spec \Z$: arithmetic Riemann-Roch. 


\section{Thom-Porteous Formula}


Let $E$ and $F$ be vector bundles , $c(E-F)=c(E)/c(F)$---calculated using the geometric series $\frac{1}{1+x}=1-x+x^2-x^3+\cdots$:
	\[
	\dfrac{c(E)}{c(F)}= \dfrac{1+c_1(E)+c_2(E)+\cdots}{1+c_1(E)+c_2(F)} = (1+c_1(E)+c_2(E)+\cdots)(1- (c_1(F)+c_2(F)+\cdots) + (c_1(F)+\cdots)^2 \cdots
	\]
Given $c_0,c_1,\ldots=c$, $\Delta_q^{(p)}(c)$ is the determinant of the $p \times p$ metric $[c_{q+i-j}]$, $1 \leq i, j \leq p$. The $i,j$ entry is $c_{q+j-i}$.

Let $\sigma: E \to F$ be a morphism of vector bundles over a purely $n$-dimensional scheme $X$, $e=\rank E$, $f=\rank F$, and let $k \leq \min(e,f)$. For each $p \in X$, $\sigma$ induces a linear transformation $\sigma_p: E_p=k^E \to F_p=k^f$. Define $D_k(r)=\{p \in X \colon \rank \sigma_p \leq k \}$. We make $D_k(\sigma)$ into a closed subscheme of $X$ as follows: cover $X$ with open sets on which both $E$ and $F$ are trivial. 
	\[
	\begin{tikzcd}
	U_i \times k^e \arrow{rr} \arrow{dr} & & U_i \times k^f \arrow{dl} \\
	& U_i & 
	\end{tikzcd}
	\]
$\sigma\big|_{U_i}$ will be given by a $e \times f$ matrix whose entries are regular functions on $U_i$. Now $D_k(\sigma) \cap U_i$ are the zeros of the $(k+1) \times (k+1)$ minors of that matrix that gives the scheme structure. 

\begin{thmm}[Thom-Porteous Formula, Modified]
Let $X$ be a scheme of pure dimension $n$. Set $m=n-(e-k)(f-k)$. Let $E$ be a vector bundle of rank $f$. Set $\mathcal{D}_k(\sigma):= \Delta^{(e-k)}(c(F-E)) \cap [X] \in A_m(X)$. Each irreducible component of $D_k(\sigma)$ has codimension at most $(e-k)(f-k)$ in $X$. If $\codim(D_k(\sigma),X)=(e-k)(f-k)$, then $\mathcal{D}_k(\sigma)$ can be represented as a positive cycle whose support is $D_k(\sigma)$. Moreover, if $\codim(D_k(\sigma),X)=(e-k)(f-k)$ and $X$ is Cohen-Macaulay then $D_k(\sigma)$ is Cohen-Macaulay and $\mathcal{D}_k(\sigma)=[D_k(\sigma)]$. 
\end{thmm}


\begin{ex}
Let $E$ be a vector bundle of rank $r$ on a quasi-projective variety $X$ over an algebraically closed field. 
\begin{enumerate}[(i)]
\item For a suitable line bundle $L$, there are sections $s_1,\ldots,s_{r+1}$ of $E \otimes L$ such that for any $p \leq r$, $D_p=\{x \in X \colon s_1(x), \ldots, s_{r-p+1}(x) \text{ are dependent}\}$ has pure dimension $p$ in $X$ or is empty, and $[D_p]=c_p(E \otimes L) \cap [X]$. 
\item The classes $c_p(E)$ are determined from $c_i(E \otimes L)$ and $c_1(L)$ by the formula 
	\[
	c_p(E)= \sum_{i=0}^p (-1)^{p-i} \binom{r-i}{p-i} c_1(L)^{p-i} c_i(E\otimes L).
	\]
``The Chern classes are the obstructions to trivializing $E$.'' 
\end{enumerate}
\end{ex}
